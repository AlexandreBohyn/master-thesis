\makeglossaries

% Acronyms
\newacronym{eppn}{EPPN}{European Phenotyping Platforms Network}
\newacronym{EU}{EU}{European Union}
\newacronym{QTL}{QTL}{Quantitative Trait Loci}
\newacronym{PLS}{PLS}{Penalized Least Squares}
\newacronym{SpATS}{SpATS}{Spatial Analysis using Tensor Splines}
\newacronym{inra}{INRA}{Institut National de Recherche Agronomimque (FR) - \textit{National Institute of Agronomical Research}}
\newacronym{VIF}{VIF}{Variance Inflation Factor}
\newacronym{DOE}{DOE}{Design of experiment}

% Definitions
\newglossaryentry{phenotype}{
	name={Phenotype},
	description={Profiling of the structures and functions associated with allelic variants, at the scale of cells, organs, whole plants and canopy}}
	
\newglossaryentry{biostress}{
	name={Biotic stress},
	description={A stress that is caused in plants due to damage instigated by other living organisms, including fungi, bacteria, viruses, parasites, weeds, insects, and other native or cultivated plants.}}
	
\newglossaryentry{abiostress}{
	name={Abiotic stress},
	description={Negative impacts on plants caused by external non-living environmental factors.}}

\newglossaryentry{quantiTL}{
	name={Quantitative trait loci},
	description={Regions of the genome containing one or more genes, associated with variations of a quantitative trait.}
}

\newglossaryentry{screening}{
	name={Screening experiment},
	description={An experiment designed to evaluate the significance of factors and factor-interactions in a model. The factor are usually two-level factor because they are either present in the model or not.}
}

\newglossaryentry{Rs_exp}{
	name={Response surface experiment},
	description={An experiment designed to find the optimal settings for the factors. In this experiment, the significant factors of the model are already determined.}
}

\newglossaryentry{genome}{
	name={Genome},
	description={All of the genes of an individual or organism. All the information that codes for the phenotype that a species displays.}
}

\newglossaryentry{genotype}{
	name={Genotype},
	description={Genetic information, present in the DNA, for a particular trait.}
}

\newglossaryentry{germplasm}{
	name={Germplasm},
	description={Germplasm are living genetic resources such as seeds or tissues that are maintained for the purpose of animal and plant breeding, preservation, and other research uses.}
}

\newglossaryentry{runs}{
	name={Experimental run},
	description={When an experiment is performed a single time in its entirety.}
}

\newglossaryentry{rank}{
	name={Rank (of a matrix)},
	description={The rank of a matrix is the maximum number of linearly independent row-vectors}
}

% DEFINITIONS TO INCLUDE:
% - phenotype
% - phenome
% - genotype
% - genome
% - phenomics
% - locus: A locus (plural loci) in genetics is a fixed position on a chromosome, like the position of a gene or a marker (genetic marker).[ (doi:10.1016/0307-4412(95)90659-2)

%\newglossaryentry{}{
%	name={},
%	description={}
%}

\makeglossaries
