\RequirePackage{fix-cm}
\documentclass[12pt,a4paper,twoside,table]{report}
\usepackage[utf8]{inputenc}
\usepackage[T3,T1]{fontenc}
%\usepackage[english]{babel}

%% FORMAT %%

%Page settings
\topmargin -10mm
\textwidth 160truemm
\textheight 240truemm
\oddsidemargin 0mm
\evensidemargin 0mm

%Header and footer settings
\usepackage{fancyhdr} %Header package
\pagestyle{fancy}
\fancyhead{}
\fancyhead[LE]{\leftmark}
\fancyhead[RO]{\rightmark}
\fancyfoot{}
\fancyfoot[LE,RO]{\textbf{\thepage}}
\renewcommand{\headrulewidth}{0.3pt} %Set header rule width 

%% FONTS %%

\usepackage{textcomp} % nice greek alphabet
\usepackage{pifont}   % Dingbats
\usepackage{booktabs}
\renewcommand\familydefault{\sfdefault} %Define the font
\usepackage{csquotes} %to avoid probelms with babel in quotes

%% MATH & EQUATIONS %%

\usepackage{amssymb,amsthm}
\usepackage{amsmath}
\usepackage{bm} % bold math
\usepackage{amsfonts} %for bold font in math environement
\usepackage{wasysym} %to have the diameter symbol
\numberwithin{equation}{subsection} %allow equation numbering to be divided by ubsection numbering
\renewcommand{\theequation}{\thesubsection\arabic{equation}} %avoid double dot in the numbering of equation using subsection numbering

%% FIGURES %%

\usepackage{float}
\usepackage[section]{placeins} 
\usepackage[font=small,labelfont={bf}]{caption} %To get caption on subfigures
\usepackage{subcaption} %To get subfigures
\usepackage{rotating} %To have landscape figures
\graphicspath{ {figures/} } %Where the figures are stored
\usepackage{tikz}
\usetikzlibrary{matrix,arrows,decorations.pathmorphing}
\usetikzlibrary{automata,positioning,calc}
\usetikzlibrary{arrows.meta, quotes}

%% TABLES %%
%Command and package for tables
\usepackage{multirow}
\usepackage{array}
\usepackage{threeparttable} % footnotes in tables
\usepackage{footnote}
\usepackage{cellspace}
    \setlength\cellspacetoplimit{4pt}
    \setlength\cellspacebottomlimit{4pt}
\newcolumntype{C}[1]{>{\centering\arraybackslash}p{#1}} %new type of column
\usepackage{arydshln} %dashed lines in tables

%% APPENDIX %% 

\usepackage[toc,page]{appendix}

%% COVER %%

\usepackage{xcolor} %the table option allows us to alternate coloring in table rows
\usepackage{graphicx,textpos}
\usepackage{helvet}

%Colors for the title page
\definecolor{green}{RGB}{172,196,0}
\definecolor{bluetitle}{RGB}{29,141,176}
\definecolor{blueaff}{RGB}{0,0,128}
\definecolor{blueline}{RGB}{82,189,236}
\setlength{\TPHorizModule}{1mm}
\setlength{\TPVertModule}{1mm}

%% BIBLIOGRAPHY %%
\usepackage[backend=biber,
style=nature,
citestyle=authoryear-comp,
doi=false,
isbn=false,
url=false,
uniquename=init,
giveninits]{biblatex}
\addbibresource{references.bib}
\usepackage[breaklinks,hidelinks]{hyperref} %For url in bibliography

%% EXTRA PACKAGES %%
\usepackage{afterpage} %To get a blank page
\usepackage{lipsum} %To test display
\usepackage{todonotes} %To add to-do notes in the text
\usepackage{enumitem} %To have descriptive lists
\usepackage{chemformula} %To write chemical formulas more easily
\usepackage{pdfpages} % to include pdf as whole pages in the appendix


%% NEW COMMANDS/SYMBOLS %%

%% Marker symbols for the caption
\usetikzlibrary{shapes}
% Empty square
\newcommand{\emptysquare}{\raisebox{0.5pt}{\tikz{\node[draw,scale=0.4,regular polygon, regular polygon sides=4,fill=none](){};}}}

% Blackline
\newcommand{\blackline}{\raisebox{2pt}{\tikz{\draw[-,black,solid,line width = 0.9pt](0,0) -- (5mm,0);}}}

%% New command for horizontal lines in matrices in TikZ diagrams
% \mhline[<style>]{<matrix name>}{<column number below of the line>}{<number of columns in a row>}
\newcommand\mhline[4][]{%
  \node[fit=(#2-#3-1),inner sep=0pt,outer sep=0pt](R){};
  \foreach \i in {1,...,#4}\node[fit=(R) (#2-#3-\i),inner sep=0pt,outer sep=0pt](R){};
  \draw[#1] (R.north -| #2.west) -- (R.north -| #2.east);
}

%% Create the inverted breve in math mode %%
\DeclareSymbolFont{tipa}{T3}{cmr}{m}{n}
\DeclareMathAccent{\invbreve}{\mathalpha}{tipa}{16}

%% Command and package to make a "formal" environment to have note inside the text.
\usepackage{framed} % for defining framed environment
\newenvironment{formal}{
    \def\FrameCommand{{\color{red}\vrule width 2pt}\hspace{2pt}}
    \MakeFramed{\advance\hsize-\width}
    \vspace{2pt}\noindent\hspace{-7pt}\vspace{3pt}
    }{\vspace{3pt}\endMakeFramed}

%% Get a clear page (for the cover page) %%
\newcommand\blankpage{%
    \null
    \thispagestyle{empty}%
    \addtocounter{page}{-1}%
    \newpage} 

%% GLOSSARY %%

%Remove the dot at the end of each abbreviation - nopostdot
%Remove page number at the end of each abbreviation nonumberlist
%Prevent abbreviation grouping - nogroupskip
%Add "List of Abbreviations" to Table of Content - toc

%acronym,nopostdot,nogroupskip,nonumberlist,toc,automake
\usepackage[acronym]{glossaries} %Load glossaries package
