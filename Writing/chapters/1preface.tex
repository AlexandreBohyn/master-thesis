%Preface
The basis for this thesis stemmed from my desire to join knowledge from my previous master in bio-engineering to my newly acquired statistical analysis skills. Instead of choosing a pre-made thesis topic, I decided to reach out to different professors both in my current and former university, to create something unique. My goal was to find an interesting topic that would spark my interest and motivate me to work hard.
Students in the master of statistics are often given datasets to analyse in the framework of their thesis. However, I wanted to give more than just a context to my dataset. For this reason, I decided to set up my own experiment, harvest the data and analyse them with a goal in mind. It took more time than expected and I had a few bumps along the road, but it was worth it. It allowed me to see the bigger pictures of scientific research. I realized that data analysis and statistical thinking are everywhere, from the pre-conception of an experiment to the detailed analysis of its results.\\

In truth, I could not have achieved this thesis without the essential help of some key people. First of all, I want to sincerely thank Professor Peter Goos from KULeuven. He agreed to be my promoter when I approached him with a personal thesis topic, and ever since he has been guiding me throughout this project, giving key insights and great support. He helped me conceiving the experiment, and advised me wisely on data analysis and interpretation. Then, I would like to thank Professor Xavier Draye from UCLouvain. Not only did he agree to let me use the phenotyping platform in the UCLouvain greenhouses, he especially helped me setting up and carrying out the experiment correctly. Finally, I would like to thank Emilie Millet and Professor Fred van Euwijk from the University of Wageningen. They were at the base of this project and were the one that helped me creating and defining the goals of this master thesis. Emilie and Xavier were also my main advisors for the phenotyping experiment. Without their support and the collaboration of the INRA in Montpellier, the data used in this thesis simply would not be here today. Finally I could not have achieved this work without the strong and constant support of my family. Thank you all for your unwavering support. 