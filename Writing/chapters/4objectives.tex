%Objectives
As is the case with every programme unit, at the end of a master’s thesis a list of objectives should be reached. These objectives are closely related to the attainment targets of the study programme. Therefore, the link with the rest of the programme is very important: which skills do students have when they start their master’s thesis? What can be expected?
For the master’s thesis the emphasis lies on competences of students to actively contribute to scientific research. At the Faculty of Science the following specific objectives are aimed for and evaluated:

    Formulating research questions with the help of the supervisor, and elaborating the research.
    Acquiring information independently and assessing its relevance for answering the research questions.
    Acquiring attitude to work on scientific research in a team (with colleague master students, PhD students, …).
    Learning to communicate in a scientific language through collaboration with fellow students and researchers.
    Following up and analysing developments in the chosen area, through training and by making contact with the current research in one of the areas.
    Using adequate experimental or theoretical methods and techniques.
    Critically analysing the results and their interpretation.
    Reporting and presenting the original results in an orderly way and placing the open questions in the right perspective. Linking techniques and results from literature as well as actual research and future research lines with the research.
     

In order to reach these objectives students should receive enough guidance without restraining their autonomy.