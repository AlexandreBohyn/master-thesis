\section{Plant phenotyping}
The terms phenotype and phenotyping are often interpreted in diverse ways between authors and between studies. In order to avoid any confusion, it is important to define these concepts clearly\footnote{An extensive list of all the needed definitions is available in the glossary in the forepart of this thesis.}.
plant phenotyping is defined as the identification of effects on the phenotype (i.e., the plant appearance and performance) as a result of genotype differences (i.e., differences in the genetic code) and the environmental conditions to which a plant has been exposed \parencite{houle_phenomics:_2010,fiorani_future_2013}. In this thesis, we refer to phenotyping more precisely as the set of methodologies and protocols used to measure plant growth, architecture, and composition with a certain accuracy and precision at different scales of organization \parencite{fiorani_future_2013}.\\
Plant phenotyping is an important tool to address and understand plant environment interaction and its translation into
application in crop management practices, effects of biostimulants, microbial communities, etc$\ldots$
\parencite{pieruschka2019plant}. 
In our current society, food security is a rising issue and genetic crop improvement is seen as a solution to deal with this issue. While genetic editing techniques and genome mapping technologies are blooming, they depend on a similar improvement in phenotyping, since they are key to analyse plant responses to environmental characterization.
In recent years, high-throughput and high-resolution phenotyping tools have made impressive progress and can now help relieving the current phenotyping bottleneck \parencite{tardieu_plant_2017,fiorani_future_2013,furbank_phenomics_2011}.
Different phenotyping platforms are emerging around the world. They range from high-precision platforms for cell and organ characterization \parencite{vargas_mapping_2006} to multi-environment networks of fields, exploiting remote sensing \parencite{virlet_field_2017}. At all scales, phenotyping facilities display spatial heterogeneity that needs to be separated from the genetic signal. For example, the spatial variability of incident light raises up to 30\% between pots within a glasshouse or a growth chamber \parencite{cabrera-bosquet_high-throughput_2016}. There are also evidences of microclimate variations in greenhouses experiments \parencite{brien_accounting_2013}. Therefore, correcting for spatial trends and using appropriate experimental designs is crucial for a precise estimation of genetic effects. Hence, the existing design and modelling theory for field experiments needs to be adapted for the phenotyping platforms.\\

\section{Experimental design in field trials}
Experimental field trials in agriculture have always been affected by soil heterogeneity. As \textcite{van_es_1.2_2002} explains, soil is a continuum with variability on multiple scales. 
The heterogeneity is as much affected by microscopic interactions as by field-sized effects. 
Therefore, agricultural trials have always heavily relied on randomisation, blocking and replication to account for spatial variability and remove bias from the estimation of the treatment effects \parencite{atkinson_one_2001}. 
For randomisation to be truly effective, stationarity of the mean and spatial independence assumptions need to be verified. Several studies have proven that it is rare that both these assumptions hold in field trials \parencite{davidoff_method_1986,nielsen_spatial_1973,iqbal_spatial_2005}. 
Moreover, \textcite{van_es_spatial_1993} showed that even randomized designs can still be problematic for experiments with large numbers of treatments and low numbers of replications in the presence of spatial autocorrelation. A new class of design has been proposed involving the use of replicated plots for a percentage of the test lines: the “p-rep” designs \parencite{cullis_design_2006,velazco_modelling_2017}.
Local field trends can influence groups of treatments in specific blocks. As a solution, several authors \parencite{watson_spatial_2000,fagroud_accounting_2002} have suggested considering the spatial trends and autocorrelation structures when creating the design, by using prior soil information, but taking into consideration spatial variability in the design of a trial not only require previous information on the plot but is often costly and cumbersome. 
Furthermore, in practice, most experimenters have neither the capacity to implement advanced designs (in terms of computation power and statistical training), nor the capacity to analyse them. 
Finally, \textcite{van_es_spatially-balanced_2007} showed that completely randomized (43 \% in greenhouse trials) and random block designs (70 \% in field trials) are still widely used.
Considering this global issue, finding and using an appropriate design is complex task.

\section{Spatial modelling for field trials}

In order to increase the precision of the estimation of genetic effects, experimental designs need to be complemented with appropriate models of analysis. Mixed model analyses using the autoregressive ($AR1$) functions   \parencite{cullis_spatial_1991} have become a standard strategy in field trials. 
However, \textcite{piepho_problems_2015} recently discussed several issues with this model and have therefore proposed the use of the linear variance ($LV$) model \parencite{williams_use_1988} instead. More specifically, \textcite{piepho_linear_2010} have proposed a revised version of this model, augmenting it into two dimensions ($AR1 \times AR1$). The main novelty resides in the addition of spatial components to a classic rows-columns model. Recently, 
\textcite{rodriguez-alvarez_correcting_2018} introduced a novel spatial model that adjusts for both global and local trends simultaneously: the SpATS model (Spatial Analysis of field Trials with Splines). The new spatial method makes use of penalized splines \parencite{eilers_flexible_1996} to estimate a bivariate smooth function over the rows and columns of a plot. Using the work of \textcite{lee_efficient_2013,lee_hwang_smoothing_2010,lee_p-spline_2011} the spatial variability is characterized using tensor products of two-dimensional P-splines \parencite{dierckx_curve_1995} and decomposed in a PS-ANOVA system. By exploiting the similarities between P-splines and mixed models \parencite{currie_flexible_2002,durban_adjusting_2001, wand_smoothing_2003}, the P-splines are expressed as a mixed model, which allows the use of classical mixed-model software but also the use of additional random and fixed effects to the model to better capture the variation along the 2-dimensional field.
It has already been tested on simulated data \parencite{rodriguez-alvarez_correcting_2018} and previous field trials data \parencite{lado_increased_2013} and showed promising results.\\

As \textcite{wilkinson_nearest_1983} and \textcite{gilmour_accounting_1997} highlight, in field trials data modelling, three main sources of spatial variations need to be accounted for:
\begin{description}
    \item[Stationary variations] 
    	\footnote{\textcite{risser_nonstationary_2016} defines a stationary process as follows:\\
	    Let $C$ be a spatial covariance function, it is said to be stationary if the features of $C$ do not depend on spatial 
	    location. More formally, a process $\{Y(\mathbf{s}) : \mathbf{s} \in G\}$ is said to be second-order stationary (or 
	    weakly 
	    stationary) if the following two properties hold:
		    \begin{enumerate}
		        \item $E[Y(\mathbf{s})]=E[Y(\mathbf{s}+\mathbf{h})]=c$ for some constant $c$ and 
		        \item $C(\mathbf{s}, \mathbf{s}+\mathbf{h})=C(\mathbf{0}, \mathbf{h})$
		        for some spatial lag $\mathbf{h} \in \mathcal{R}^{d}$.
		    \end{enumerate}
	    }
    Large scale trends across the field (e.g. fertility trend, depth of soil, moisture)
    \item[Non-stationary variations] Also natural variations but localized on part of the field (e.g. patch of soil moisture)
    \item[Extraneous variations] Variations unrelated to a natural process, often due to the way the field is prepared (e.g. 
    tillage, sowing practices, etc$\ldots$)
\end{description}
A part of these variations can be attributed to systematic effects, e.g. sowing or planting, other to random effects such as fertility trends. While systematic effects can easily be modelled using factors and row-columns attributes, it is not case the case for random spatial variation. They are harder to model because there are no covariates to relate it to. Since the spatial variation has both random and systematic components, it makes sense to use the mixed model framework.\\
There are two main approaches to model spatial trends: one based on spatial variance-covariance structures; and the other based on smoothing techniques. The SpATS model uses a smooth bivariate surface to model both the global and local trends, while accounting for the extraneous variations by using extra random and/or fixed coefficients. Models using spatial covariance structure (such as the $AR\times AR$) model the global trends using functions of the spatial coordinates (both linear trends and smoothing splines), while the local trends are estimated with the use of spatially dependant error term (thus the reason why these models use spatial covariance structure) and the extraneous trends are managed similarly to the SpATS model. In this thesis, the data extracted from the phenotyping platform are modelled using these 2 different models.

\section{Thesis objectives}
This master thesis falls within the scope of the second activity of the European project EPPN2020\footnote{European Plant Phenotyping Network 2020 \url{https://eppn2020.plant-phenotyping.eu/}}. It is a research infrastructure project funded by  Horizon 2020, that will provide access to 31 key plant phenotyping installations. It defines three research activities: (1) novel technologies and methods for environmental and plant measurements, (2) innovative design and analysis of phenotyping experiments across multiple
platforms and (3) a European plant phenotyping information system. The project revolves around data acquisition, data analysis and data networking, so that every platform uses common, standardized practices and analysis protocols, that have been tested for robustness and quality.\\

The main goal was to assess the utility of statistical designs and mixed models to identify and correct for spatial trends (heterogeneity) in an aeroponic root installation at UCLouvain (Louvain-la-neuve). The idea is to set up an experiment in this installation using different genotypes (plant varieties) and a custom experimental design to account for possible complex environmental variations. It will be created using JMP, taking into account the specifities of the platform and the number of genotypes used. This approach allows the design to fit the experiment properly and avoids having to use a pre-made design, not optimal for the experiment.
%Its efficiency will be tested against classical pre-made designs, such as the Randomized Complete Block (RCB) and alpha-lattice designs, and the best one will be applied on the phenotyping platform. 
After data collection and image analysis, two different models will be used to model the spatial variability and to assess the quality of spatial prediction. The first one is a two-dimensional version of the linear variance model, revised by \textcite{piepho_linear_2010}. The second one is the SpATS (Spatial Analysis using Tensor product of Splines) model, recently created by \textcite{rodriguez-alvarez_correcting_2018}. 
The two models will be compared in term of their ability to estimate genotypic effects and to quantify spatial variability. These comparisons will be made using classical indicators (RMSE, $\ldots$) and other indicators, specific to spatial models for field trials \parencite{oakey_joint_2006}.\\

The experiment took place in February in the UCLouvain greenhouses. The installation consists of two aeroponic tanks of 495 plants located in a 64 m$^2$ greenhouse. Plants are held on strips, 5 plants per strip, 99 strips per tank. The specificity of 
the platform is that the plant rotate constantly so that their root system can be photographed every two hours.
The experiment lasted 3 weeks, after which the plants became too large for the platform. The experiment included two tanks. In the first one, plants constantly moved to be pictured every 2 hours (usual set-up on this platform). In the second one, plants  moved twice or three times a day to be pictured. This allowed comparing the effect of moving versus non-moving plants, which is a feature often available in the phenotyping platforms but poorly evaluated so far.\\

Since the UCLouvain platform focuses on the analysis of the root system, the main variable of interest in the experiment is the overall growth of the root system of each plant. 
An experiment generates a large amount of data in a raw, image-based format (approximately 200K images). 
Images allow temporal decoupling, provide a condensed set of information and are multi-dimensional (2D usually, but 3D scanning platforms are developping \parencite{mooney_developing_2012}). 
A lot of tools are available for data analysis in phenotyping platforms \parencite{lobet_online_2013}. This makes the choice complicated for an external user, especially since most of these softwares are designed for a single specific purpose. Another challenge in root system architecture (RSA) characterisation is the inherent complexity of the system. Different techniques have been developed to best characterize the RSA in a cost-efficient way \parencite{pound_rootnav:_2013,lobet_novel_2013}. 
Scientists of the UCLouvain platform have developed pipelines\footnote{Here, pipelines are defined as computer programs designed to analyse raw data from phenotyping platforms.} that allow easy processing of the images captured in the platform to extract quantitative root architecture information for the spatial models \parencite{lobet_novel_2013,lobet_novel_2011}.\\

This thesis can be summarised in four main points: create an appropriate experimental design for a phenotyping experiment, analyse data from a high-throughput platform, comparing the efficiency of various spatial models to correct for heterogeneous and non-linear spatial trends and developing the appropriate R scripts.