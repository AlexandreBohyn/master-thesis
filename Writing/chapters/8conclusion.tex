%Conclusion

This thesis aimed to evaluate the efficiency of spatial modelling techniques on the UCLouvain phenotyping platform. More precisely, we wanted to assess the differences between a still and a moving growing tank in the platform, and to estimate the genotypic effects as precisely as possible with the use of spatial models for field trials. In parallel, we also wanted to evaluate the performance of a classical spatial model against the SpATS model, a new technique, that recently showed promising results in the framework of field trials analysis.\\

To achieve this goal, we created a custom experimental design, fitted to the platform and set up a phenotyping experiment. The results indicated a strong difference in yield between the tanks. This was confirmed by the results of both models, proving this effect to be highly significant. This indicated that the moving environment of the phenotyping platform was more profitable to plant growth, because they were less sensitive to local spatial variations. Moreover, this moving environment allowed a more precise and continuous characterization of the plants. The results also hinted at an important genotypic effect. Some genotypes expressed much lower yield than other, even more so, in the moving environment where the spatial influence is supposed to be less pronounced. These genotype features were also retrieved by both spatial models, that gave similar results.\\

In terms of performance, both spatial models gave similar results. They exposed the complexity of the spatial trends present in the data and were both able to account for it with the same magnitude. However, these models mainly differed in their parametrization and interpretation. The SpATS model was more versatile because it accounted for both local and global trends in a single process and did not require a stepwise selection process for the best model, like most classical linear models do. The interpretation of the parameters is also simpler with SpATS. The concept of effective dimensions introduced with the model allowed to easily see the contribution of each spatial component to the whole fitted spatial surface. Conversely the standard spatial model gave interpretable parameters, but the interpretation might not be as straightforward. Moreover, the SpATS model was also more flexible to pick up spatial patterns in highly heterogeneous environments, whereas the standard models might consider this heterogeneity as random error. Overall the SpATS model seemed to be a better choice for spatial modelling in the context of this experiment. \\

However, spatial modelling may not be the best tool to analyse data from a moving phenotyping platform. The very concept of row and column position is hard to translate to plants that are constantly shifting positions. Even if some spatial trends remain influential with those changing locations, the relative distance between plants is changing, making it hard to model spatial correlation correctly. Depending on the goal of the platform, the modelling strategy could be rethought to incorporate these challenges in a meaningful way.\\

This reflection opens interesting perspective for future research. If the identification of spatial trends is secondary, and the evaluation of the genotypes is the main objective, it could be interesting to compare the results of a generalized linear model (GLM) against the spatial models used here. Especially given the specificity of the moving environment. Furthermore, we could even more capitalize on the scan of the plants realized during the experiment. The root pictures could reveal interesting information that were not present, or not detected in the weights of the plants. The main perspective remains to incorporate the evolution of root growth in the model. Since several pictures of each plant were taken during the experiment, we have longitudinal data of each plant. This untapped potential could reveal even more insights on the difference between genotypes.\\

In conclusion, this thesis had both an agronomical interest, to estimate the genotypes effects and a statistical interest, to compare the efficiency of different spatial models. From a phenotyping point of view, this thesis has shown that this type of platform are helpful in identifying competitive genotypes. They could help relieving the current bottleneck and help the future of plant breeding.  From the statistical angle, both models yielded similar results, but SpATS was the superior in terms of parametrization. This new model seems to be a very promising tool for spatial analysis of field trials on phenotyping platform.

