%Introduction

In the recent years, food security has grown to become a worldwide concern due to the challenges imposed on farmers by climate change. 
In this context, genetic improvement to increase crop resistance to external stresses, is seen as solution to generate novel traits in plants and help fight those threats \parencite{tester2010breeding}. 
In the last decades, genetic editing and sequencing techniques have greatly improved, leading to a high-throughput of genetic data and the associated data analysis techniques \parencite{schiml2016revolutionizing}. 
However, in order to capitalize on those discoveries, similar advancements need to be made regarding plant phenotyping. 
Even though many breakthrough regarding sensors and imaging techniques are made, plant phenotyping still remains a challenge \parencite{furbank_phenomics_2011}. Due to the extreme sensitivity of plants to their growing environment, a large amount of the research is dedicated to study plants' variations that's unrelated to their genetic traits. Researchers are setting up high-throughput platforms, where thousands of plants are sequentially analysed, to be able to quantify the influence of environmental conditions on plants' growth \parencite{tardieu_plant_2017}.\\

In parallel, plant breeding trials often involve a large number of genotypes and large areas where spatial variation is likely to be an obstacle to reliable genetic prediction. To account for complex spatial variations, researchers often use spatial analysis methods that model the correlation between neighbouring plots \parencite{velazco_modelling_2017}. There is a large array of different methods such as nearest neighbours analysis \parencite{wilkinson_nearest_1983}, spatial covariance structures \parencite{gilmour_accounting_1997,piepho_linear_2010}, polynomials models \parencite{federer1998recovery} or smoothing splines \parencite{durban_adjusting_2001}. However, for these approaches to be efficient, they require a complementing experimental design. Over the years, several types of complex experimental designs have been created and tested in field trials \parencite{yates_comparative_1939,patterson_efficiency_1983,cullis_design_2006}. However, even though those designs allow good correction for field trends, they are rarely used in practice because of their complexity and the specific cases to which they apply. This is even more relevant for phenotyping platforms, where the experimental set-up is rarely similar to the one of a field.\\ 

With this background in mind, the motivation behind this thesis was to design and conduct an experiment in a phenotyping platform and then analyse the results using different spatial models. We created an optimal design, fitted to the platform, and applied in experiment with different genotypes of maize. We then harvested the plants and analysed the weights using different spatial models, to compare their efficiency on this given platform. The fact that we over-viewed each step of the experiment from the germination of the seeds to the analysis of the results took a large amount of time but allowed more control over the different phases of the process. 
On this platform, plants are constantly moving to be characterized, which renders the spatial trend quantification complicated. Even though this feature is often available in platforms, it is poorly evaluated. Therefore, in order to make this thesis relevant we wanted to evaluate the impact of constant movements on the plants growth.\\ 


To summarize, this thesis can be divided in two sections. The first one is about creating en experiment adapted to the platform's characteristics in order to evaluate the impact of movements on plant growth, as well as the effect of each genotype. The second part is about fitting different spatial models in order to see which one is better able to capture, and explain, the spatial trends in the data. We do not aim at creating new spatial models or revolutionizing experimental design, but rather to apply those concepts on a practical case. The following sections are a litterature review of the topic; then a development of the material and methods used (that is the experimental procedure and the spatial models); a presentation and discussion of the results; and finally a conclusion to see if and how the goals, presented in this introduction, were met in the thesis. 
